\documentclass[
	% -- opções da classe memoir --
	article,			% indica que é um artigo acadêmico
	11pt,				% tamanho da fonte
	oneside,			% para impressão apenas no verso. Oposto a twoside
	a4paper,			% tamanho do papel. 
	% -- opções da classe abntex2 --
	%chapter=TITLE,		% títulos de capítulos convertidos em letras maiúsculas
	%section=TITLE,		% títulos de seções convertidos em letras maiúsculas
	%subsection=TITLE,	% títulos de subseções convertidos em letras maiúsculas
	%subsubsection=TITLE % títulos de subsubseções convertidos em letras maiúsculas
	% -- opções do pacote babel --
	english,			% idioma adicional para hifenização
	brazil,				% o último idioma é o principal do documento
	sumario=tradicional
	]{abntex2}
\usepackage[utf8]{inputenc}


\title{Poluição Atmosférica e seus efeitos colaterais em SP}


\date{May 2022}

\begin{document}

\maketitle

\section*{Integrantes do Grupo}
\begin{itemize}
    \item Pedro Almeida e Camacho
    \item Wellynton Gonçalves
    \item Rubens Prada 
    \item Manuel Vitor
    \item Israel de Assis
\end{itemize}


\section{resumo}
A questão principal deste artigo é discutir a poluição atmosférica no centro de SP e como ela afeta crianças e idosos, com a justificativa de que este tipo de poluição afeta diretamente o sistema respiratório dos humanos e com isso pode levar diversas pessoas a contrair doenças respiratórias onde pode se criar casos criticos em crianças e idosos. A metodologia utilizada para confecção do trabalho foi apenas pesquisas bibliográficas, o artigo busca demonstrar algumas propostas que podem trazer benefícios em relação a redução desses poluentes assim diminuindo junto a isso o numero de crianças e idosos que podem contrair doenças respiratórias devido a poluição atmosférica.
\subsection{Palavras-Chave}
Poluição atmosférica, idosos e crianças, doenças respiratórias, São Paulo, poluição.



\section{Introdução}
Durante os últimos anos vivemos um duro período de pandemia, causado pelo vírus da covid-19, que nos levou a tomar diversas medidas para conter o avanço da doenças, tais elas como o isolamento total e a quarentena da população, com isso um estudo realizado pela UFLA mostrou que os poluentes atmosférico tiveram uma grande redução na cidade de São Paulo  com isso a ideia de estudar sobre como esses poluentes afetam a saúde das pessoas veio a tona para o grupo tendo em vista que diversas medidas de restrições foram flexibilizadas nos últimos meses com isso aumentando muito a circulação de veículos, o fluxo de pessoas na rua e o funcionamento de empresas e fabricas, podendo assim alterar o nível de poluição atmosférica em SP assim podendo voltar a afetar a saúde de idosos e crianças que normalmente são os mais afetados.

Com isso o problema que será discutido neste artigo é como a poluição atmosférica provoca diversas doenças para nossa população como a mesma pode ser um grande problema para porcentagens especifica de pessoas como crianças e idosos que são os que mais são afetados.  

Dessa maneira neste artigo buscamos trazer soluções que possam reduzir o impacto da poluição atmosférica em idosos e crianças do mesmo jeito que diminuir este tipo de poluição em Sp, tendo em vista que com as mudanças de restrição contra a covid-19 existem chances da emissão de toxinas a atmosfera aumentarem assim aumentando o numero de pessoas sendo afetadas pela poluição atmosférica .

\section{Objetivo}
O objetivo do artigo é trazer alternativas que busquem reduzir os impactos causados pela poluição atmosférica em crianças e idosos em Sp.

\section{Metodologia}
A metodologia utilizada pelo grupo baseada no pensamento computacional onde é decomposto o problema proposto assim reconhecido padrões ou seja o porque esse problema acontece e seus padrões apos abstraímos o conteúdo de maneia a manter apenas o que é útil para cumprir os objetivos do trabalho, assim chegamos a ultima parte o algorítimo que é o que será realizado para concluir estes objetivos.
\section{Discussão}

As crianças e os idosos são os que mais sofrem com a poluição atmosférica no centro de SP. 
A criança por ainda não ter o sistema imunológico totalmente desenvolvido, e os mais velhos por começarem a ter problemas no mecanismo de defesa do próprio corpo. 
Boa parte da sujeira presente no ar é filtrada pelo próprio corpo humano. Os pêlos dentro do nariz, por exemplo, servem para segurar a poluição. Mas existe um tipo de poeira muito fina que nem esses pelos conseguem frear. Essas partículas mínimas de poluição vão parar no pulmão, o que faz com que os habitantes das grandes cidades adoeçam mais.

Muitos estudos mostram uma associação positiva entre mortalidade e morbidade por problemas respiratórios em crianças e idosos no centro de SP.  A poluição atmosférica tem sido associada a aumentos de internações e de mortalidade, tanto por doenças respiratórias quanto por doenças cardiovasculares. Dentre os principais poluentes do ar, podemos citar a fumaça, partículas inaláveis, dióxido de enxofre, ozônio, dióxido de nitrogênio e monóxido de carbono. Essas substâncias podem causar sérios danos à saúde do homem. O monóxido de carbono, por exemplo, diminui a capacidade do sangue de transportar oxigênio pelo corpo, podendo causar hipóxia tecidual. Já o ozônio possui papel oxidante e citotóxico, podendo causar irritação nos olhos e diminuição da capacidade pulmonar, por exemplo. O dióxido de enxofre relaciona-se com irritações nas vias aéreas superiores, assim como o dióxido de nitrogênio. Esse último também pode provocar danos graves aos pulmões. E todas essas substâncias estão concentradas no centro de SP.

A qualidade de vida das pessoas é afetada diretamente pela emissão de gases dos automóveis, principalmente no centro de SP. Isso acontece devido à poluição causada pelas substâncias tóxicas emitidas pelos veículos. Segundo o relatório mais recente da Companhia Ambiental do Estado de São Paulo (Cetesb), com dados de 2008 a 2022, a frota de veículos é responsável na Grande São Paulo por 98\% das emissões de monóxido de carbono (CO), 97\% de hidrocarbonetos (HC), 9\% de óxido de nitrogênio (NOx), 40\% de material particulado e 33\% de óxido de enxofre (SOx). Todos são materiais que poluem o meio ambiente.

Desde o início das medidas de restrição devido à pandemia de Covid-19, estudos têm sido realizados levando em consideração situações atípicas desse momento de urgência e emergência, dentre eles, a relação da diminuição do tráfego de veículos e melhoria da poluição ambiental.  Infelizmente a redução total da poluição é impossível, porém é possível reduzir um pouco e evitar um aumento. Pensando na possibilidade, o grupo decidiu seguir  o exemplo da avenida paulista, abrindo o centro para o público e proibindo automóveis se locomoverem no centro, e de segunda a terça apenas transporte público e meios de transporte que não emitem CO2.  Também em Fortalecer a ciência de dados por trás das políticas de qualidade do ar, principalmente pela ampliação e pelo aperfeiçoamento do sistema de monitoramento atmosférico nacional, priorizando áreas críticas e uso de novas tecnologias.
Podemos concluir que diante os fatos apresentados, à poluição causada pelas substâncias tóxicas emitidas pelos veículos e transportes públicos, mostram uma associação positiva entre mortalidade e morbidade por problemas respiratórios em crianças e idosos no centro de SP.


\section{Consideração finais}
Infelizmente a poluição atmosférica é um problema seríssimo que pode afetar diretamente de forma mais impactante idosos e crianças, com isso o objetivo do trabalho era buscar formas apropriadas de se reduzir este tipo de poluição em Sp e como também evitar contrair doenças que essa poluição causa, porem reduzir uma poluição por completo é impossível, sendo assim foram propostas ideias para diminuir a emissão de gases tóxicos a atmosférica e como você como ser humano pode realizar para evitar poluir nossa atmosfera assim como evitar contrair doenças que ela pode nos causar.  
\end{document}